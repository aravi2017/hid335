\documentclass[sigconf]{acmart}

\usepackage{hyperref}

\usepackage{endfloat}
\renewcommand{\efloatseparator}{\mbox{}} % no new page between figures

\usepackage{booktabs} % For formal tables

\settopmatter{printacmref=false} % Removes citation information below abstract
\renewcommand\footnotetextcopyrightpermission[1]{} % removes footnote with conference information in first column
\pagestyle{plain} % removes running headers

\begin{document}
\title{Big Data Analytics, Data Mining, and Health Informatics: 
Data Mining Social Media to Track Epidemics}
\author{Sean M. Shiverick}
% \orcid{1234-5678-9012}
\affiliation{%
  \institution{Indiana University}
  \streetaddress{10214 Galway}
  \city{Bloomington} 
  \state{Indiana} 
  \postcode{12345}
}
\email{smshiver@indiana.edu}


\begin{abstract}
This paper reviews research literature on data mining of social media to 
track the outbreak of epidemics at the population level. Data mining of 
search queries and twitter posts have been used to monitor seasonal influenza
as an early detection of epidemics to supplement official sources of public 
health information. This work shows how big data analytics has applications 
for public health informatics. Limitations of using social media to predict 
epidemics are discussed.
\end{abstract}

\keywords{Data Mining, Social Media, Population Health Informatics}


\maketitle

\section{Introduction}

In the information age, \textit{Big Data} offers great promise: to fuel innovation, 
generate new revenue streams, and transform society \cite{gupta15}. Can the 
potential of big data be harnessed for the greater good, to prevent disease, 
and improve health? Seasonal influenza epidemics are a major concern for public 
health, resulting each year in an estimated 250,000 to 500,000 deaths worldwide \cite{who17}. This paper explores big data in public health informatics, 
specifically the use of data mining to track epidemics and the spread of 
contagious disease \cite{hay13}. Epidemic spreading is a complex phenomenon, 
however, based on contact networks between individuals and distributed by 
transportation networks \cite{Colizza06}. Can these approaches be extended to 
monitor other epidemics such as the opioid crisi in North America \cite{volkow14}?


\subsection{Public Health Informatics}

The field of Health Informatics is generating huge amounts of data at a rapid pace,
from MRI imaging data, electronic medical records (EMRs), clinical research data, to
population-level data. This review focuses on population level from social media to 
provide insights about epidemics and pandemics \cite{hay13, herland14}. Big data is 
an ambiguous term, that lacks a single unified definition, but is often described in 
terms of: Volume, Velocity, Variety, Veracity, and Value \cite{demchenko12}. Trying 
to track an epidemic in real-time from multitudes of incoming web posts (e.g., 
Twitter), involves a high volume of data coming in at high velocity \cite{lamb13, 
paul14}. In order to be of any use, diverse and often messy raw data has to be sifted
through and effectively organized for further analysis. The issue of Veracity raises 
the questions of how reliable are social media posts for predicting real life events.
What is the relationship between data mined from social media to biological events 
such as the spreading of disease and contagion? The question of Value evaluates the 
quality of the data obtain in relation to intended outcomes to be attained. There may
be some question as to the quality of data collected from social media; however, a
review of the literature suggests that mining information from social media can 
produce data of high value. An important challenge for making sense of big data 
is developing analytic tools adequate to handle large volumes of data in real time.

\subsection{Data Mining Social Media}

Health Informatics research can be considered from two levels: where the data 
is collected, and the research questions being addressed. Research on social 
media can yield data on a range of issues related to public health, including: 
spatiotemporal information of disease outbreaks, real-time tracking of infectious
diseases, global distributions of various diseases, and search queries on medical 
questions that people might have. The questions of interest in the current review 
are: first, \textit{Can search query data be used to accurately track epidemics in 
a population (in real-time)?};  and second, \textit{Can Twitter post data be used 
to track an epidemic across a population?} A limitation of social media data is 
that, as mentioned above, although it has high Volume, Velocity, and Variety, it 
could be unreliable, resulting in both low Veracity and Value \cite{hay14, lazer14}. 
However, useful data can be extracted through data mining and analytic techniques. 

\subsection{Using Search Queries to Track Epidemics}

\subsubsection{Tracking Epidemics Using {\itshape Google} Search Terms in U.S.}

Seasonal influenza is an acute viral infection that spreads easily from person to 
person, circulates across the world, affecting people of every age. Traditional 
flu monitoring estimates from the U.S. Center of Disease Control and Prevention 
(CDC) based on physician reports of patients with influenza-like illness (ILI) are 
released weekly, but generally with a one to two week delay. In an effort to improve 
on early detection of season influenza, a team of researchers developed an automated 
method to analyze Google search queries to track ILI terms from historical logs 
between 2003 and 2008, using 50 million most popular searches, and CDC historical 
data \cite{ginsburg09}. The Google Flu Trends (GFT; https://www.google.org/flutrends)
model sought to find the probability that a given search query is related to an ILI 
of a patient visiting a physician in the same region. Ginsberg and colleagues used 
a feature selection method to narrow  the 50 million most popular search queries down
to 45. These top queries showed connections to influenza symptoms, complications, 
remedies, that were consistent with searches by individuals with influenza. The 
researchers based their estimates of the current level of weekly influenza based on 
the correlation of the relative frequency of search queries and the percentage of 
physician visits with patients presenting influenza-like symptoms. Thus, analyzing 
high volume Google search queries was used to predict ILI epidemics in real time for 
areas with a large population of web users, and provide information to the public in 
a matter of days to help physicians and hospitals prepare and respond to the 
outbreak. 

\subsubsection{Tracking H1N1 Epidemic Using {\itshape Baidu} Searches in China}

In China, researchers used a similar method to monitor influenza activity based on
internet search query data from Baidu ({\it baidu.com}) compared to influenza case counts 
from the Chinese Ministry of Health (MOH) between 2009 to 2012 (during H1N1 epidemic) 
\cite{yuan13}. This approach was based on four parts (1) selecting keyword terms related 
to influenza, (2) filtering keywords unrelated to flu epidemics, (3) defining weights 
and composite search index, and (4) fitting regression model with keyword index to 
influenza case data. In the process of filtering only 40 of 94 keywords were correlated 
with the case data, and only 8 of these 40 keywords were used as the optimal set in the 
composite search index. As expected, the search index captured seasonal variation of 
influenza epidemics in the Winter and Spring, indicating a good predictor for tracking 
influenza activity in China. The regression model accounted for 95 percent of the
variability in influenza case data (ICD), and the model was validated for a test period 
in 2012. The mean absolute percent error rate was less than 11 percent for the validation
test data. This research yields additional evidence that novel approaches using big data 
can provide early indicators of epidemic activity that supplement official public health 
information sources, rather than replacing them. A limitation acknowledged by the authors
is the relatively small initial number of keyword search terms used compared to the 
Google Flu Trends (GFT) project cite{ginsberg09}. Another limitation of using search 
query data is that, although the keywords selected in this model performed well at 
capturing temporal trends in the H1N1 epidemic, the same keywords may not reflect the 
trend of an influenza epidemic at a future time. The authors also noted the lack of 
internet access in rural areas, which underscores the fact that effective tracking of 
epidemics based on search queries relies on internet access. Furthermore, caution should 
be used when evaluating predictions based on correlational data, as causation cannot be 
inferred from correlation.

\subsection{Using {\itshape Twitter} API Data to Track Epidemics}

Twitter is a free online social networking and microblogging service, where users can send 
and read messages of 140 characters (i.e., ``tweets''). As of 2017, Twitter has more than 
320 million monthly active users (67 million in U.S.), with an estimated 500 million tweets 
posted per day ({\it https://about.twitter.com}). Twitter users share their perspectives and 
reactions on a wide range of topics, approximately 80 percent from handheld mobile devices, 
acting as ``sensors'' of events in real time cite{achrekar12}. The Twitter stream provides 
a rich data source for tracking or forecasting general sentiment, political attitudes, 
linguistic variation, detecting earthquakes, and disease surveillance. The large volume of 
users provides a high likelihood that ILI epidemic information is posted; however, Twitter 
post data is noisy and perhaps unreliable insofar as it can be difficult to differentiate 
posts about the flu based on instances of concerned awareness ({\it ``I am worried about 
the swine flu epidemic!''}) versus actual infection, ({\it ``Robbie might have swine flu. 
I am worried.''})\cite{lamb13}. Despite the noise in Twitter data from much useless chatter, 
useful information be found. 

\subsubsection{Using Twitter to Track Disease Activity and Public Concern in the 
U.S. during the Influenza A H1N1 Pandemic {signorini11}}

In a 2011 study, researchers searched through post data from Twitter`s streaming 
application programmer`s interface (API) during the H1N1 epidemic (October 2009 to May 2010) 
across spatiotemporal areas of the U.S. to predict weekly ILI levels {signorini11}. Tweets 
were sifted according to keywords related to H1N1 (flu, swine, influenza) and additional 
terms about vaccines, side effects, and/or vaccine shortages. The first data set consisted 
of 951,697 tweets containing influenza-related keywords from 334,840,972 tweets extracted 
between April 29 and June 1, 2009 (results reported as a percentage of observed tweets). 
These tweets represent just over 1 percent of the sample tweet volume, and this percentage 
declined rapidly over time as the number of reported H1N1 cases increased. In the U.S. 
surveillance programs track reported influenza-like illness (ILI) seasonally (October to 
May), monitoring the total number of patients seen along with the number with ILIs reported. 
Quantitative estimates of ILI values based on the Twitter stream were constructed support 
vector regression and leave-one-out cross-validation. Weekly ILI values were estimated using 
a model trained on roughly 1 million influenza-related tweets obtained between October, 2009 
to May 2010. Point estimates of national ILI values produced by the system were good with 
an average error of 0.28 percent. A regional model, based on significantly fewer tweets, 
approximated the epidemic curve for CDC region 2 (New York, New Jersey) as reported by the 
ILI data, but the estimate was less precise with an average error of 0.37 percent. In terms 
of public interest, Twitter users' interest in antiviral drugs such dropped as official 
disease reports indicated most influenza cases were relatively mild, even as the number of 
cases was increasing. In addition, interest in hand hygiene and face masks was associated 
with public health messages from CDC. A limitation of the study is that only a limited 
number of search terms and one prediction method was used. An important question is whether 
the results could be improved using broader search terms and other prediction models. 


\subsubsection{Twitter Improves Seasonal Influenza Prediction. \cite{achrekar12}}

In a 2012 study, researchers constructed a system using an online social network Crawler 
bot to retrieve tweets by keywords ({\it flu, H1N1, swine flu}), geospatial location, 
relative keyword frequency , and CDC ILI reports \cite{achrekar12}. The Social Network 
Enabled Flu Trends (SNEFT) network continuously monitored tweets and profile details of 
the Twitter users who commented on flu keywords (starting October 2009), to detect and 
track the spread of ILI epidemics. The correlation between flu related tweets and ILI 
was  very high between 2009–2010 (r=0.98) during the H1N1 outbreak, but the correlation 
dropped substantially for 2010-2011 (r=0.47), suggesting that noisy tweets became more 
prominent as H1N1 was less of an issue. To reduce noise, text classification (SVMs) was 
trained on a dataset of 25,000 tweets to determine whether a tweet was related to a flu 
event or not; data cleansing removed multiple tweets posted by the same user during a 
single bout with the same illness. These methods improved the correlations between the 
actual CDC percent of ILI-related visits compared to Twitter data alone for October 2010 
to May 2011, across the U.S. and its subregions. The authors reported that Twitter data 
alone had higher prediction rates toward the beginning and end of the flu season, and 
during an epidemic. The results showed that Twitter data can be used to detect and possibly
track Influenza like epidemics in real time; It would also be useful to extend the results 
outside the U.S. to different populations  in other countries cite{yuan13}. The results 
provide a method for predicting ongoing ILI epidemics based on Twitter data, in real 
time, with relatively low error, up to 1-2 weeks earlier than the CDC. 


\section{Conclusions}

\subsubsection{Limitations of Using Search Queries and Social Media Data to Track Epidemics}

The research reviewed above shows how data mining search queries and twitter posts for 
ILI related information provides an early detection signal to supplement existing epidemic 
monitoring systems and may help improve public health responses and prevention. However, 
as described above, there are several limitations in using these methods for predicting 
epidemics: First, internet access and Twitter usage is not uniform by geographical region; 
urban areas have higher density of network connections than rural areas cite{yuan13}, and 
coastal regions of the U.S. (CA, NY) produced more tweets per person than Midwestern states 
(or Europe) \cite{achrekar12}. Thus, performance of seasonal influenza predictions models 
may be limited to regions with high internet access and where tweets are more frequent. 
Second, exact demographic information about the Twitter population is not easy to estimate 
(or unknown) and the demographic of internet users does not represent characteristics of 
the general population. If we consider that outbreaks such as swine flu or avian flu 
originated at points of contact between humans and domesticated animals in agricultural 
areas, then internet searches or Twitter posts would provide limited information to predict 
epidemic spreading in the larger population. Third, though promising, the results of this 
research are based on correlations between often noisy internet search queries or Twitter 
posts and physician reports of ILI compiled by official governmental sources. We should be 
cautious in evaluating predictions about serious health concerns such as epidemics or 
pandemics based on correlational evidence as the data does not support causal inferences 
which are the basis of scientific claims. 

 
\subsubsection{Can Social Media Prediction Models be Generalized to Track Other Epidemics?}






\bibliographystyle{ACM-Reference-Format}
\bibliography{report} 

\end{document}
