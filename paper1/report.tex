\documentclass[sigconf]{acmart}

\usepackage{hyperref}

\usepackage{endfloat}
\renewcommand{\efloatseparator}{\mbox{}} % no new page between figures

\usepackage{booktabs} % For formal tables

\settopmatter{printacmref=false} % Removes citation information below abstract
\renewcommand\footnotetextcopyrightpermission[1]{} % removes footnote with conference information in first column
\pagestyle{plain} % removes running headers

\begin{document}
\title{Big Data Analytics, Data Mining, Health Informatics: 
Data Mining Social Media to Track Epidemics}
\author{Sean M. Shiverick}
% \orcid{1234-5678-9012}
\affiliation{%
  \institution{Indiana University}
  \streetaddress{10214 Galway}
  \city{Bloomington} 
  \state{Indiana} 
  \postcode{12345}
}
\email{smshiver@indiana.edu}


\begin{abstract}
This paper is a literature review of research on data mining of social media 
to predict ourbreaks of epdiemics at the population level. This work shows how 
big data analytics has applications for public health informatics.
\end{abstract}

\keywords{Data Mining, Social Media, Population Health Informatics}

\maketitle

\section{Introduction}

In the tech era, \textit{Big Data} offers great promise: to fuel innovation, 
generate new revenue streams, and transform society. Along with heightened 
expectations about about the potential big data is a certain amount of hype 
\cite{gupta15}; big data has been described as the \textit{new oil} of the 
information age.  The basic task for data scientists, sifting through gargantuan 
volumes of data in search of actionable insights, is highly automated with the 
use of algorithms. Conventional adages such as, \textit{finding the needle in 
a haystack} have been turned on its head, as data scientists are \textit{turning 
hay into needles}, essentially searching for gold in mountains of raw data ore. 
Can the potential of big data may be harnessed for the greater good, to prevent 
disease, to improve health? Seasonal influenza epidemics are a major concern 
for public health, resulting each year in an estimated 250,000 to 500,000 deaths 
worldwide \cite{who17}. This paper explores big data in public health informatics, 
specifically the use of data mining to track epidemics and the spread of contagious 
disease \cite{hay13}. Epidemic spreading is a complex phenomenon, however, based 
on contact networks between individuals and distributed by transportations networks 
\cite{Colizza06}. Limitations of using data mining to predict epidemics are 
discussed. Also considered is how these approaches may be extended to monitor 
the opioid epidemic in North America \cite{volkow14}.


\section{Big Data: Population Health Informatics}

As an ambiguous term, big data lacks a single, unified definition; however, it 
is often described in terms of: Volume, Velocity, Variety, Veracity, and Value 
\cite{demchenko12}. The field of Health Informatics is generating huge amounts 
of data at a rapid pace, from MRI imaging data, electronic medical records (EMRs), 
clinical research data, to population-level data. This review focuses on data 
collected at the population level from social media to provide insights about 
epidemics and pandemics  \cite{hay13, herland14}. Trying to track an epidemic in 
real-time from multitudes of incoming web posts (e.g., Twitter), involves a high 
volume of data coming in at high velocity \cite{lamb13, paul14}. In order to be 
of any use, diverse and often messy raw data has to be sifted through and effectively 
organized for further analysis. The issue of Veracity raises the questions of how 
reliable are social media posts for predicting real life events. What is the 
relationship between data mined from social media to biological events such as 
the spreading of disease and contagion? The question of Value evaluates the quality 
of the data obtain in relation to intended outcomes to be attained. There may be 
some question as to the quality of data collected from social media; however, a 
review of the literature suggests that mining information from social media can 
produce data of high value. An important challenge for making sense of big data 
is developing analytic tools adequate to handle large volumes of data in real time.


\subsection{Social Media Data Mining}

Health Informatics research can be considered from two levels: where the data 
is collected, and the research questions being addressed. This paper focuses on 
population level data to address questions about epidemics. Research on social 
media can yield data on a range of issues related to public health, including: 
spatiotemporal information of disease outbreaks, real-time tracking of infectious
diseases, global distributions of various diseases, and search queries on medical 
questions that people might have. The questions of interest in the current review 
are: first, \textit{Can search query data be used to accurately track epidemics in 
a population (in real-time)?};  and second, \textit{Can Twitter post data be used 
to track an epidemic across a population?} A limitation of social media data is 
that, as mentioned above, although it has high Volume, Velocity, and Variety, it 
could be unreliable, resulting in both low Veracity and Value \cite{hay14, lazer14}. 
However, useful data can be extracted through data mining and analytic techniques. 


\subsection{Using Search Query Data to Track Epidemics}

\subsubsection{Tracking Epidemics Using {\itshape Google} Search Terms in U.S.}

Seasonal influenza is an acute viral infection that spreads easily from person to 
person , circulates across the world, affecting people of every age. Traditional 
flu monitoring estimates from the U.S. Center of Disease Control and Prevention 
(CDC) based on physician reports of patients with influenza-like illness (ILI) are 
released weekly, but generally with a one to two week delay. In an effort to improve 
on early detection of season influenza, a team of researchers developed an automated 
method to analyze Google search queries to track ILI terms from historical logs 
between 2003 and 2008, using 50 million most popular searches, and CDC historical 
data \cite{ginsburg09}. The Google Flu Trends (GFT; https://www.google.org/flutrends) 
model sought to find the probability that a given search query is related to an ILI 
of a patient visiting a physician in the same region. Ginsberg and colleagues used 
a feature selection method to narrow  the 50 million most popular search queries down 
to 45. These top queries showed connections to influenza symptoms, complications, 
remedies, that were consistent with searches by individuals with influenza. The 
researchers based their estimates of the current level of weekly influenza based on 
the correlation of the relative frequency of search queries and the percentage of 
physician visits with patients presenting influenza-like symptoms. Thus, analyzing 
high volume Google search queries was used to predict ILI epidemics in real time for 
areas with a large population of web users, and provide information to the public in 
a matter of days to help physicians and hospitals prepare and respond to the outbreak. 

\subsubsection{Tracking H1N1 Epidemic Using {\itshape Baidu} Searches in China}

Researchers in China used a similar method to monitor influenza activity based 
on internet search query data from Baidu (baidu.com) compared to influenza case 
counts China’s Ministry of Health (MOH) between 2009 to 2012 (during the H1N1 
epidemic) \cite{yuan13}. This approach was based on four parts (1) selecting 
keyword terms related to influenza, (2) filtering keywords unrelated to flu 
epidemics, (3) defining weights and composite search index, and (4) fitting 
regression model with keyword index to influenza case data. In the process of 
filtering only 40 of 94 keywords were correlated with the case data, and only 8 
of these 40 keywords were used as the optimal set in the composite search index. 
As expected, the search index captured seasonal variation of influenza epidemics 
in the Winter and Spring, indicating a good predictor for tracking influenza activity
in China.  Their regression model accounted for 95% of the variability in influenza
case data (ICD), and the model was validated for a test period in 2012. The mean 
absolute percent error rate was less than 11% for the validation test data. A 
limitation acknowledged by the authors is the relatively small initial number 
of keyword search terms compared to those reported by Ginsberg et al. with the 
Google Flu Trends model. This study contributes to the evidence showing that novel
approaches using big data can help provide early indicators for disease outbreak, 
and can provide information on epidemic activity as a supplement to official sources, 
rather than a replacement. Another limitation of using search query data is that, 
although the keywords selected for use in this model performed well at capturing 
the temporal trend in the epidemic, the same keywords may not reflect the trend of 
an influenza epidemic at a future time. The authors also noted the lack of internet 
access in rural areas, which underscores the fact that effective tracking of epidemics 
based on search queries relies on internet access. Furthermore, “correlation does 
not imply causation”, and caution should be taken in evaluating predictions based 
on correlational data.    


\subsection{Using {\itshape Twitter} API Data to Track Epidemics}

The details of the construction of the \texttt{.bib} file are beyond
the scope of this sample document, but more information can be found
in the \textit{Author's Guide}, and exhaustive details in the
\textit{\LaTeX\ User's Guide} by Lamport~\shortcite{Lamport:LaTeX}.

This article shows only the plainest form of the citation command,
using \texttt{{\char'134}cite}.

We use jabref to manage all citations. A paper without managing a bib
file will be returned without review. in the bibtex file all urls are
added to rfernces with the {\it url} filed. They are not to be
included in the {\it howpublished} or {\it note} field. 



\section{Conclusions}

This paragraph will end the body of this sample document.  Remember
that you might still have Acknowledgments or Appendices; brief samples
of these follow.  There is still the Bibliography to deal with; and we
will make a disclaimer about that here: with the exception of the
reference to the \LaTeX\ book, the citations in this paper are to
articles which have nothing to do with the present subject and are
used as examples only \cite{Paul2014}.


\begin{acks}

  The author would like to thank Dr. Gregor Von Laszewski for providing the 
  \LaTex template for the paper and \JabRef bibliography for the references. 
  Thanks also to the Assistant Instructors for their helpful feedback.

\end{acks}

\bibliographystyle{ACM-Reference-Format}
\bibliography{report} 

\end{document}
