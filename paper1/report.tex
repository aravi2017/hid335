\documentclass[sigconf]{acmart}

\usepackage{hyperref}

\usepackage{endfloat}
\renewcommand{\efloatseparator}{\mbox{}} % no new page between figures

\usepackage{booktabs} % For formal tables

\settopmatter{printacmref=false} % Removes citation information below abstract
\renewcommand\footnotetextcopyrightpermission[1]{} % removes footnote with conference information in first column
\pagestyle{plain} % removes running headers

\begin{document}
\title{Big Data Analytics, Data Mining, Health Informatics: 
Data Mining Social Media to Track Epidemics}
\author{Sean M. Shiverick}
% \orcid{1234-5678-9012}
\affiliation{%
  \institution{Indiana University}
  \streetaddress{10214 Galway}
  \city{Bloomington} 
  \state{Indiana} 
  \postcode{12345}
}
\email{smshiver@indiana.edu}


\begin{abstract}
This paper is a literature review of research on data mining of social media 
to predict ourbreaks of epdiemics at the population level. This work shows how 
big data analytics has applications for public health informatics.
\end{abstract}

\keywords{Data Mining, Social Media, Population Health Informatics}

\maketitle

\section{Introduction}

In the tech era, \textit{Big Data} offers great promise: to fuel innovation, 
generate new revenue streams, and transform society. Along with heightened 
expectations about about the potential big data is a certain amount of hype 
\cite{gupta15}; big data has been described as the \textit{new oil} of the 
information age.  The basic task for data scientists, sifting through gargantuan 
volumes of data in search of actionable insights, is highly automated with the 
use of algorithms. Conventional adages such as, \textit{finding the needle in 
a haystack} are turned on its head, as enterprising data scientists are turning 
hay into needles, essentially searching for gold amidst mountains of ore. How 
can the potential of big data may be harnessed for the greater good, to prevent 
disease, to improve health? This paper explores big data in public health 
informatics, specifically how data mining of social media has been used to track 
epidemics and the spread of conagious disease \cite{hay13}. Epidemic spreading 
is complex, based on networks of contact between individuals and distributed by 
transportations networks \cite{vespignani15}, and limitations of using social 
media to predict epidemics are discussed. Also considered is how these approaches 
may be extended to monitor the opioid epidemic in North America \cite{volkow14}.


\section{Big Data: Population Health Informatics}

Big data is an ambiguous term that lacks a single, unified definition; a useful 
approach has been to define big data in terms of: Volume, Velocity, Variety, 
Veracity, and Value \cite{demchenko12}. The field of Health Informatics is 
generating huge amounts of data at a very rapid pace, from genomic data, patient 
tissue samples, MRI imaging data, electronic medical records (EMRs), clinical
research data, up to population-level data. This review focuses on public data 
collected at the population level from social media that may provide insights about 
health concerns such as epidemic spreading and pandemics  \cite{hay13, herland14}. 
Trying to track an epidemic in real-time from multitudes of incoming web posts 
(e.g., Twitter), involves a high volume of data coming in at high velocity 
\cite{lamb13, paul14}. The complex nature of this data reflects the wide variety of 
information collected, and in order to be of any use, the data has to be sifted 
through and effectively organized for further analysis.The issue of Veracity raises 
questions of utmost importance: How reliable are social media posts for predicting 
real life events? What is the relationship between data mined from social media to 
biological phenomenon such as the spreading and disease and contagion? The question 
of Value evaluates the quality of the data obtain in relation to intended outcomes 
to be attained. There may be some question as to the quality of data collected from 
social media; however, a review of the literature suggests that mining information 
from social media can produce data of high value. An important challenge for making 
sense of big data is developing data mining and analytic methods to handle large 
volumes of rapidly growing data. 

\subsection{Social Media Data Mining}

Health Informatics research can be considered from two levels: where the data is 
collected, and the research questions being addressed. This paper focuses on 
population level data to address questions about epidemics. Research on social media 
can yield data on a range of issues related to public health, including: spatio-
temporal information of disease outbreaks, real-time tracking of infectious diseases, 
global distributions of various diseases, and search queries on medical questions
that people might have. The questions of interest in the current review are: first, 
\textit{Can search query data be used to accurately track epidemics in a population 
(in real-time)?};  and second, \textit{Can Twitter post data be used to track an 
epidemic across a population?} A limitation of social media data is that, as 
mentioned above, although it has high Volume, Velocity, and Variety, it could be 
unreliable, resulting in both low Veracity and Value \cite{hay14, lazer14}. 
However, useful data can be extracted through data mining and analytic techniques. 

\subsection{Using Search Query Data to Track Epidemics}

Seasonal influenza is an acute viral infection that spreads easily from person 
to person , circulates across the world, affecting people of every age. Worldwide, 
seasonal influenza epidemics are a major public health concern, resulting in an 
estimated 3 to 5 million cases of severe illness and 250,000 to 500,000 deaths 
each year \cite{who16}. In the U.S., traditional flu monitoring estimates from
the Center of Disease Control and Prevention (CDC) based on physician reports of
patients with influenza-like illness (ILI) are released weekly, but generally with 
a one to two week delay.

\subsubsection{Tracking Epidemics Using {\itshape Google} Search Terms in U.S.}

A team of researchers developed an automated method to analyze Google search 
queries to track ILI terms from historical logs between 2003 and 2008, using 50 
million most popular searches, and CDC historical data \cite{ginsberg09}. The 
Google Flu Trends (GFT) model sought to find the probability that a given search 
query is related to an ILI of a patient visiting a physician in the same region. 
Ginsberg and colleagues used a feature selection method to narrow down the 50 
million most popular search queries to 45. The top queries showed connections to 
influenza symptoms, complications, remedies, that werew presumably consistent with 
searches by individuals with influenza. This method of analyzing high volume of 
Google search queries was designed to predict development of ILI epidemics in 
areas with a large population of web users, and provide information to the public 
in a matter of days rather than weeks. Ultimately, the project sought to provide 
actionable information about ILI epidemics in real time to help physicians and 
hospital prepare to stop the spread of the disease. Despite the high correlations, 

how accurate was the prediction provided by GFT?




\subsubsection{Tracking H1N1 Epidemic Using Search Queries on {\itshape Baidu} in China }

Again, in either environment, you can use any of the symbols
and structures available in \LaTeX\@; this section will just
give a couple of examples of display equations in context.
First, consider the equation, shown as an inline equation above:


\subsection{Using {\itshape Twitter} API Data to Track Epidemics}

The details of the construction of the \texttt{.bib} file are beyond
the scope of this sample document, but more information can be found
in the \textit{Author's Guide}, and exhaustive details in the
\textit{\LaTeX\ User's Guide} by Lamport~\shortcite{Lamport:LaTeX}.

This article shows only the plainest form of the citation command,
using \texttt{{\char'134}cite}.

We use jabref to manage all citations. A paper without managing a bib
file will be returned without review. in the bibtex file all urls are
added to rfernces with the {\it url} filed. They are not to be
included in the {\it howpublished} or {\it note} field. 



\section{Conclusions}

This paragraph will end the body of this sample document.  Remember
that you might still have Acknowledgments or Appendices; brief samples
of these follow.  There is still the Bibliography to deal with; and we
will make a disclaimer about that here: with the exception of the
reference to the \LaTeX\ book, the citations in this paper are to
articles which have nothing to do with the present subject and are
used as examples only \cite{Paul2014}.


\begin{acks}

  The author would like to thank Dr. Gregor Von Laszewski for providing the 
  \LaTex template for the paper and \JabRef bibliography for the references. 
  Thanks also to the Assistant Instructors for their helpful feedback.

\end{acks}

\bibliographystyle{ACM-Reference-Format}
\bibliography{report} 

\end{document}
