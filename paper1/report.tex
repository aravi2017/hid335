\documentclass[sigconf]{acmart}

\usepackage{hyperref}

\usepackage{endfloat}
\renewcommand{\efloatseparator}{\mbox{}} % no new page between figures

\usepackage{booktabs} % For formal tables

\settopmatter{printacmref=false} % Removes citation information below abstract
\renewcommand\footnotetextcopyrightpermission[1]{} % removes footnote with conference information in first column
\pagestyle{plain} % removes running headers

\begin{document}
\title{Big Data Analytics, Data Mining, and Public Health Informatics: 
Using Data Mining of Social Media to Track Epidemics}
\author{Sean M. Shiverick}
% \orcid{1234-5678-9012}
\affiliation{%
  \institution{Indiana University-Bloomington}
}
\email{smshiver@indiana.edu}

\begin{abstract}

Data mining of internet search queries and social media for influenza related keywords 
has been used to track seasonal influenza and correlates highly with official reports 
of `infuenza-like-illness' (ILI). Efforts to monitor epidemics using big data analytics 
can provide early detection that supplements existing systems of disease surveillance. 
A review of the literature shows that data extracted from social media has applications 
for public health informatics. Prediction models based on social media work best in 
areas with a high degree of internet access.  

\end{abstract}

\keywords{i523, HID335, Data Mining, Social Media, Public Health Informatics}

\maketitle

\section{Introduction}

In the information age, \textit{Big Data} offers great promise to fuel innovation, 
generate new revenue streams, and transform society \cite{gupta15}. Can the 
potential of big data be harnessed for the greater good, to prevent disease 
and improve health? Seasonal influenza epidemics are a major public health concern, 
that each year result in an estimated 250,000 to 500,000 deaths worldwide 
\cite{who17}. This paper explores big data in public health informatics, 
specifically reviewing research on data mining to track epidemics and the spread 
of contagious disease \cite{hay13}. Can these approaches be extended to monitor 
other epidemics such as the opioid crisis in North America? \cite{volkow14}
Epidemic spreading is a complex phenomenon based on contact networks between 
individuals and distributed by transportation networks \cite{Colizza06}. Some 
question remains as to whether prediction models based on social networking 
platforms can be generalized to other epidemics. Limitations of using social 
media data to predict epidemics are discussed.



\subsection{Public Health Informatics}

The field of Health Informatics is generating huge amounts of data at a rapid pace,
from MRI imaging data, electronic medical records (EMRs), clinical research data, to
population-level data. This review focuses on population data from search queries and 
social media to provide insights about epidemics and pandemics \cite{hay13, herland14}. 
Big data is an ambiguous term that lacks a single unified definition, but is often 
described in terms of {\it Volume, Velocity, Variety, Veracity, and Value}
\cite{demchenko12}. Trying to track an epidemic in real-time from multitudes of 
incoming web searches and posts involves a high volume of data coming in at high 
velocity \cite{lamb13, paul14}. In order to be of any use, diverse and often messy 
raw data has to be sifted through and effectively organized for further analysis. 
The issue of Veracity raises the questions of how reliable social media data are for 
predicting real life events. What is the relationship between social media data to 
biological events such as the spreading of contagion and disease? The question of 
Value evaluates the quality of the data as it pertains to intended outcomes. There 
are legitimate concerns about the quality of data obtained from the internet; 
however, the literature suggests that mining information from social media can 
produce valuable data. An important challenge for making sense of big data is 
developing analytic tools adequate to handle large volumes of data in real time.

\subsection{Data Mining Social Media}

Health Informatics research is considered from two levels: where the data is 
collected, and the research questions being addressed. Research on social media 
can yield data on a range of issues related to public health, including: 
spatiotemporal information of disease outbreaks, real-time tracking of infectious 
diseases, global distributions of various diseases, and search queries on medical 
questions that people might have \cite{herland14}. The questions of interest in the 
current review are: \textit{Can search query data be used to accurately track epidemics 
in real-time?} and, \textit{can Twitter data be used to monitor epidemics across 
different regions??}. The general idea is that increasing search query or social 
media activity is associated with an increasing interest in a given health topic. 
A limitation of social media data is that, although it has high Volume, Velocity, 
and Variety, it can be unreliable, resulting in both low Veracity and Value 
\cite{hay13, lazer14}. A review of the literature shows how useful data can be 
extracted by data mining and analytic techniques. 

\subsection{Using Search Queries to Track Epidemics}

\subsubsection{Tracking Epidemics Using {\itshape Google} Search Terms in the U.S.}

Seasonal influenza is an acute viral infection that spreads easily from person to 
person, circulates across regions, affecting people of every age. Traditional flu 
monitoring estimates from the U.S. Center of Disease Control and Prevention (CDC) 
based on physician reports of patients with ``{\it influenza-like illness}'' (ILI) are 
released weekly \cite{cdc17}, but generally with a one to two week delay. In an effort 
to improve on early detection of season influenza, a team of researchers developed an 
automated method to analyze Google search queries to track ILI terms from historical 
logs between 2003 and 2008, using 50 million most popular searches, and CDC historical 
data \cite{ginsburg09}. The {\it Google Flu Trends} (GFT, https://www.google.org/flutrends)
model sought to find the probability that a given search query is related to an ILI of a 
patient visiting a physician in the same region. GFT used a feature selection method to 
narrow  the 50 million most popular search queries down to 45. These top queries showed 
connections to influenza symptoms, complications, remedies, that were consistent with 
searches by individuals with influenza. The researchers based their estimates of the 
current level of weekly influenza based on the correlation of the relative frequency 
of search queries and the percentage of physician visits with patients presenting 
influenza-like symptoms. Thus, analyzing high volume Google search queries was used 
to predict ILI epidemics in real time for areas with a large population of web users, 
and provide information to the public in a matter of days to help physicians and 
hospitals prepare and respond to the outbreak. 

\subsubsection{Tracking H1N1 Epidemic Using {\itshape Baidu} Queries in China}

A research study in China monitored influenza activity by comparing internet search query 
data from {\it Baidu} (https://www.baidu.com) to influenza case counts from the Chinese 
Ministry of Health (MOH) between 2009 to 2012 during the H1N1 epidemic \cite{yuan13}. 
This study consisted of four parts: (i) Selecting keyword terms related to influenza, (ii) 
Filtering keywords unrelated to flu epidemics, (iii) Defining weights and composite search 
index, and (iv) Fitting a regression model with keyword index to influenza case data. In the 
process of filtering, only 40 of 94 keywords were correlated with the case data, and only 8 
of these 40 keywords were used as the optimal set in the composite search index. As expected,
the search index captured seasonal variation of influenza epidemics in the Winter and Spring,
indicating a good predictor for tracking influenza activity in China.  The regression model 
accounted for 95 percent of the variability in influenza case data (ICD), and the model was 
validated for a test period in 2012. The mean absolute percent error rate was less than 11 
percent for the validation test data. This research yields additional evidence that novel 
approaches using big data can provide early indicators of epidemic activity that supplement 
official public health information sources, rather than replacing them. A limitation 
acknowledged by the authors is the relatively small initial number of keyword search terms 
used compared to the Google Flu Trends (GFT) project \cite{ginsburg09}. Another limitation 
of using search query data is that, although the keywords selected in this model performed 
well at capturing temporal trends in the H1N1 epidemic, the same keywords may not reflect 
the trend of an influenza epidemic at a future time. The authors also noted the lack of 
internet access in rural areas, which underscores the fact that effective tracking of 
epidemics based on search queries relies on internet access. Furthermore, caution should 
be used when evaluating correlational data, as causation cannot be inferred from correlation.

\begin{table}
  \caption{Frequency of Special Characters}
  \label{tab:freq}
  \begin{tabular}{ccl}
    \toprule
    Non-English or Math&Frequency&Comments\\
    \midrule
    \O & 1 in 1,000& For Swedish names\\
    $\pi$ & 1 in 5& Common in math\\
    \$ & 4 in 5 & Used in business\\
    $\Psi^2_1$ & 1 in 40,000& Unexplained usage\\
  \bottomrule
\end{tabular}
\end{table}


\subsection{Using {\itshape Twitter} API Data to Track Epidemics}

Twitter is a free online social networking and micro-blogging service, where users can send 
and read messages of 140 characters (i.e., {\it ``tweets''}). As of 2017, Twitter has more 
than 320 million monthly active users (67 million in U.S.), with an estimated 500 million 
tweets posted per day (https://about.twitter.com). Twitter users share their perspectives and
reactions on a wide range of topics, approximately 80 percent from handheld mobile devices, 
acting as ``sensors'' of events in real time \cite{achrekar12}. The Twitter stream provides 
a rich data source for tracking or forecasting general sentiment, political attitudes, 
linguistic variation, detecting earthquakes, and disease surveillance. The large volume of 
users provides a high likelihood that ILI epidemic information is posted; however, Twitter 
post data is noisy and perhaps unreliable insofar as it can be difficult to differentiate 
posts about the flu based on instances of concerned awareness ({\it ``I am worried about 
the swine flu epidemic!''}) versus actual infection, ({\it ``Robbie might have swine flu. 
I am worried.''})\cite{lamb13}. Despite the noise in Twitter data from much useless chatter, 
useful information be obtained from mining data in the Twitter stream. 

\subsubsection{Using Twitter to Track Disease Activity and Public Concern in the 
U.S. during the H1N1 Pandemic.}

In a 2011 study, researchers searched through post data from Twitter`s streaming API 
during the H1N1 epidemic (October 2009 to May 2010) across spatiotemporal areas of the 
U.S. to predict weekly ILI levels \cite{signorini11}. Tweets were sifted according to 
keywords related to H1N1 (e.g., {\it ``flu'', ``swine'', ``influenza''}) and additional 
terms about vaccines, side effects, and/or vaccine shortages. The first data set consisted 
of 951,697 tweets containing influenza related keywords from 334,840,972 tweets extracted 
between April to June 2009 (results were reported as a percentage of observed tweets). 
These tweets represent just over 1 percent of the sample tweet volume, and this percentage 
declined rapidly over time as the number of reported H1N1 cases increased. In the U.S. 
surveillance programs track reported influenza-like illness (ILI) seasonally, from October 
to May, monitoring the total number of patients seen along with the number with ILIs 
reported. Quantitative estimates of ILI values based on the Twitter stream were analyzed 
using support vector regression (SVR) and leave-one-out cross-validation to test model 
accuracy. Weekly ILI values were estimated using a model trained on roughly 1 million
influenza-related tweets obtained between October, 2009 to May 2010. Point estimates of 
national ILI values produced by the system were good with an average error of 0.28 percent. 
A regional model, based on significantly fewer tweets, approximated the epidemic curve for 
CDC region 2 (New York, New Jersey) as reported by the ILI data, but the estimate was less 
precise with an average error of 0.37 percent. In terms of public interest, Twitter users` 
interest in antiviral drugs dropped, as official disease reports indicated most 
influenza cases were relatively mild, even as the number of cases was increasing. In 
addition, interest in hand hygiene and face masks was associated with public health 
messages from CDC. A limitation of the study is that only a limited number of search 
terms and one prediction method was used. An important question is whether the results 
could be improved using broader search terms and other prediction models. 

\begin{figure}
\includegraphics[height=1in, width=1in]{images/fly}
\caption{A sample black and white graphic
that has been resized with the \texttt{includegraphics} command.}
\end{figure}


\subsubsection{Twitter Improves Seasonal Influenza Prediction}

In a 2012 study, researchers implemented a system using an online social network (OSN)
Crawler bot to retrieve tweets by keywords (e.g., {\it ``flu'', ``H1N1'', ``swine flu''}), 
geospatial location, relative keyword frequency , and CDC ILI reports \cite{achrekar12}. 
The {\it Social Network Enabled Flu Trends} (SNEFT) network continuously monitored tweets 
and profile details of the Twitter users who commented on flu keywords (starting October 
2009), to detect and track the spread of ILI epidemics. The correlation between flu related 
tweets and ILI was very high between 2009 to 2010 (r=0.98) during the H1N1 outbreak, but 
the correlation dropped substantially for 2010-2011 (r=0.47) after the epidemic, suggesting 
that noisy tweets became more prominent as H1N1 was less of an issue. To reduce noise, text 
classification using support vector machines ({\it SVMs}) was trained on a dataset of 
25,000 tweets to determine whether a tweet was related to a flu event or not; data cleansing 
was conducted to remove multiple tweets posted by the same user during a single bout with 
the same illness. These methods improved the correlation between the Twitter data and ILI 
rates from the CDC from October 2010 to May 2011 in the U.S. (r=0.89), and Twitter data was 
correlated with ILI rates across subregions. The authors reported that Twitter data alone 
had higher prediction rates toward the beginning and end of the flu season, and during an 
epidemic. In addition, age analysis suggested Twitter data best fit the age groups of 5-24 
years and 25-49 years,  for most regions in the U.S. The results showed Twitter data can be 
used to detect and possibly predict ongoing ILI epidemics in real time with relatively 
low error, up to 1-2 weeks earlier than the CDC reportings. It would also be interesting to 
determine whether these results are generalizable outside the U.S. with different 
populations in other countries \cite{yuan13}.


\subsection{{\it Limitations} of Using Search Queries and Social Media Data to Track Epidemics}

There is some evidence that influenza forecasting models based on Twitter data performed 
better than general search query data \cite{paul14}. Google Flu Trends (GFT) algorithms 
underestimated ILI in the U.S. at the start of the H1N1 (i.e. {\it swine flu}) pandemic 
in 2009 \cite{butler13}, and over-predicted seasonal influenza in January 2013 compared 
to the CDC ILI by almost double \cite{lazer14}. As described above, there are important 
limitations in using social media data for predicting epidemics: First, internet access 
and Twitter usage is not uniform by geographical region. Urban areas have higher density 
of internet connections than rural areas \cite{yuan13}, and coastal regions of the U.S. 
(CA, NY) produced more tweets per person than Midwestern U.S. states (or Europe) 
\cite{achrekar12}. Thus, performance of seasonal influenza predictions models may be best 
applied to regions with high internet access and where tweets are more frequent. Second, 
exact demographic information about the Twitter population is not easy to estimate (or 
unknown) and the demographic of internet users does not represent characteristics of the 
general population.  Third, though promising, the results of this research are based on 
correlations between often noisy internet search queries or Twitter posts and 
physician reports of ILI compiled by official governmental sources. Caution should be  
used in evaluating predictions about serious health concerns such as epidemics or 
pandemics based on correlational evidence as the data do not support causal inferences . 

\begin{figure}
\includegraphics[height=1in, width=1in]{images/fly}
\caption{A sample black and white graphic
that has been resized with the \texttt{includegraphics} command.}
\end{figure}


\subsubsection{Can these methods be extended to survey other types of epidemics?} 

The dynamics of epidemic spreading is a 
complex phenomenon, based on contact networks of person-to-person interaction, indirect 
exposure, and transmission highways such as the {\it airline transportation network} (ATN) 
\cite{Colizza06}. Epidemics are described in terms of the proportion of the population 
infected, those yet to be infected, and the rate of transmission \cite{hethcote00}. 
In addition, the structure of the contact network can influence epidemic spreading 
\cite{pastor01}. For example, in the case of simple contagion, weak ties among 
acquaintances or infrequent associations provide shortcuts between distant nodes that 
reduce distance within the network \cite{granovetter73} and can facilitate the spread of 
disease. Furthermore, networks with ``small world'' properties have many nodes with few 
connections, but a small number of highly connected nodes that can rapidly transmit 
contagion throughout the network \cite{watts98}. Analyzing the correlation between 
Twitter posts and rate of ILI reports does not capture the complexity underlying disease 
epidemics and pandemics. By analyzing the structure of social media networks, future 
research may help to identify how points of connection online is associated with epidemic 
spreading in the external world \cite{zhu17}. There have been some efforts to address the 
opioid crisis in North America?\cite{smith16}. The emergence of new technologies, such as 
wearable biosensors \cite{carreiro15} may help improve geospatial mapping other epidemics.

\section{Conclusion}

Big data mining of social media has tremendous potential to detect trends and confirm 
observations based on real time events, providing opportunities to monitor infectious 
disease on a global level. The research reviewed above shows how search queries and 
Twitter data about ILI related information provides an early detection signal that 
supplements existing epidemic monitoring systems and may help improve public health 
responses and prevention. As described above, these approaches to predicting edidemic
may work best in areas of high population density with a high internet connectivity
and social media usage, and thus are better suited to larger urban areas than 
predicting outbreaks in rural areas. 

\bibliographystyle{ACM-Reference-Format}
\bibliography{report} 

\end{document}
